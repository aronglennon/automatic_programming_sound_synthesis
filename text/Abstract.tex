% Dissertation
% 06/28/2013
% (Use \include{filename} in diss file

\documentclass[12pt]{report} 	% report used for dissertations

\usepackage{anysize} 		% allows me to set margins
\usepackage{indentfirst}
\usepackage{apacite}		% for apa style
\usepackage{graphicx}		% for graphics
\usepackage{times}
\usepackage{mathtools}
\usepackage{textcomp}
\usepackage{amsmath}
\usepackage{afterpage}
\usepackage{setspace}
\usepackage{titlesec}
\usepackage[normalem]{ulem}

\titleformat{\chapter}[display]
  {\normalfont\fontfamily{ptm}\fontsize{14}{17}\selectfont\filcenter}{CHAPTER \thechapter}{0pt}{}

\usepackage{chngcntr}
\renewcommand\thechapter{\Roman{chapter}}

%\marginsize{1.5in}{1.5in}{1.25in}{1.25in}	% left, right, top, bottom
\usepackage[paperwidth=8.5in, paperheight=11in]{geometry}
\geometry{top = 1.25in, left = 1.5in, right = 1.5in, bottom=1.25in, footskip=0.5in}
\begin{document}
\pagenumbering{gobble}
\begin{titlepage}
\begin{center}
\begin{tabular}{llll}
  Sponsoring Committee: & Professor Robert Rowe\\
  & Professor Juan P. Bello\\
  & Professor R. Luke Dubois\\
\end{tabular}
\\
\vfill
AN ABSTRACT OF\\ 
\vspace{\baselineskip}
EVOLVING SYNTHESIS ALGORITHMS USING A MEASURE\\
\vspace{\baselineskip}
OF TIMBRAL SEQUENCE SIMILARITY\\
\vspace{\baselineskip}
\vspace{\baselineskip}
\vspace{\baselineskip}
Aron Glennon\\
\vspace{\baselineskip}
Program in Music Technology\\
Department of Music and Performing Arts Professions\\
\vfill
Submitted in partial fulfillment\\
of the requirements for the degree of\\
Doctor of Philosophy in the\\
Steinhardt School of Culture, Education, and Humanities Development\\
New York University\\
2014\\
\end{center}
\end{titlepage}

% Abstract
\newgeometry{top = 2.0in}
\doublespacing
\begin{flushleft}
\setlength{\parindent}{30pt}	% This starts every paragraph with an indent
\hspace*{\parindent}The advent of the computer has provided composers with the ability to produce absolutely any sonic material within the human range of hearing. However, this power comes with a fundamental caveat: the composer must communicate to the computer what sonic material they are interested in generating, and this process is a non-trivial task. It is primarily the case that music composition software is written to provide an interface to the composer that defines boundaries within which the composer may generate material. While these interfaces have proven effective in producing a large variety of material, the paradigm of software developers providing specific boundaries for the composer to work within is not ideal and certainly not the only possibility.

The motivation of this research is to explore a different paradigm, where the composer communicates to the software exactly what sort of timbral material they are interested in generating and the software produces an appropriate interface to help them realize that. It is based on the idea that the most natural way for the composer to communicate with a computer is via describing by example.

The proposed system accepts a short audio file with timbral content that the composer wishes to begin working with, finds a synthesis algorithm able to produce like timbres via an intelligent search algorithm, and constructs that algorithm within the Max visual programming environment for the composer to use. An appropriate intelligent search method and parameterization is proposed, which includes a definition of a timbral sequence similarity measure used to assess individual algorithms and steer the search path through synthesis algorithm space.
\end{flushleft}
\end{document}